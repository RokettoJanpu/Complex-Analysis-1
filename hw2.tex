\documentclass{article}
\usepackage{amsfonts, amsthm, amsmath, amssymb, mathtools, ulem, mathrsfs, physics, esint, siunitx, tikz-cd}
\usepackage{pdfpages, fullpage, color, microtype, cancel, textcomp, markdown, hyperref, graphicx}
\usepackage{enumitem}
\usepackage{algorithm}
\usepackage{algpseudocode}
\graphicspath{{./images/}}
\usepackage[english]{babel}
\usepackage[autostyle, english=american]{csquotes}
\MakeOuterQuote{"}
\usepackage{xparse}
\usepackage{tikz}

\usepackage{calligra}
\DeclareMathAlphabet{\mathcalligra}{T1}{calligra}{m}{n}
\DeclareFontShape{T1}{calligra}{m}{n}{<->s*[2.2]callig15}{}
\newcommand{\script}[1]{\ensuremath{\mathcalligra{#1}}}
\newcommand{\scr}{\script r}

% fonts
\def\mbb#1{\mathbb{#1}}
\def\mfk#1{\mathfrak{#1}}
\def\mbf#1{\mathbf{#1}}
\def\tbf#1{\textbf{#1}}

% common bold letters
\def\bP{\mbb{P}}
\def\bC{\mbb{C}}
\def\bH{\mbb{H}}
\def\bI{\mbb{I}}
\def\bR{\mbb{R}}
\def\bQ{\mbb{Q}}
\def\bZ{\mbb{Z}}
\def\bN{\mbb{N}}

% brackets
\newcommand{\br}[1]{\left(#1\right)}
\newcommand{\sbr}[1]{\left[#1\right]}
\newcommand{\brc}[1]{\left\{#1\right\}}
\newcommand{\lbr}[1]{\left\langle#1\right\rangle}

% vectors
\renewcommand{\i}{\hat{\imath}}
\renewcommand{\j}{\hat{\jmath}}
\renewcommand{\k}{\hat{k}}
\newcommand{\proj}[2]{\text{proj}_{#2}\br{#1}}
\newcommand{\m}[2][b]{\begin{#1matrix}#2\end{#1matrix}}
\newcommand{\arr}[3][\sbr]{#1{\begin{array}{#2}#3\end{array}}}

% misc
\NewDocumentCommand{\seq}{O{n} O{1} O{\infty} m}{\br{#4}_{{#1}={#2}}^{#3}}
\NewDocumentCommand{\app}{O{x} O{\infty}}{\xrightarrow{#1\to#2}}
\newcommand{\sm}{\setminus}
\newcommand{\sse}{\subseteq}
\renewcommand{\ss}{\subset}
\newcommand{\vn}{\varnothing}
\newcommand{\lc}{\epsilon_{ijk}}
\newcommand{\ep}{\epsilon}
\newcommand{\vp}{\varphi}
\renewcommand{\th}{\theta}
\newcommand{\cjg}[1]{\overline{#1}}
\newcommand{\inv}{^{-1}}
\DeclareMathOperator{\im}{im}
\DeclareMathOperator{\id}{id}
\newcommand{\ans}{\tbf{Ans. }}
\newcommand{\pf}{\tbf{Pf. }}
\newcommand{\imp}{\implies}
\newcommand{\impleft}{\reflectbox{$\implies$}}
\newcommand{\ck}{\frac1{4\pi\ep_0}}
\newcommand{\ckb}{4\pi\ep_0}
\newcommand{\sto}{\longrightarrow}
\DeclareMathOperator{\cl}{cl}
\DeclareMathOperator{\intt}{int}
\DeclareMathOperator{\bd}{bd}
\DeclareMathOperator{\Span}{span}
\newcommand{\floor}[1]{\left\lfloor#1\right\rfloor}
\newcommand{\ceil}[1]{\left\lceil#1\right\rceil}
\newcommand{\fxn}[5]{#1:\begin{array}{rcl}#2&\longrightarrow & #3\\[-0.5mm]#4&\longmapsto &#5\end{array}}
\newcommand{\sep}[1][.5cm]{\vspace{#1}}
\DeclareMathOperator{\card}{card}
\renewcommand{\ip}[2]{\lbr{#1,#2}}
\renewcommand{\bar}{\overline}
\DeclareMathOperator{\cis}{cis}
\DeclareMathOperator{\Arg}{Arg}
\DeclareMathOperator{\Log}{Log}

% title
\title{Complex Analysis HW 2}
\author{Ryan Chen}
\date{}
\setlength{\parindent}{0pt}


\begin{document}
	
\maketitle



\tbf{Note.} Let $f_x$ denote $\pdv{f}{x}$.
\sep



\tbf{P31.3.} \pf Write $f=u+iv$.
$$\bar{f_x} = \bar{u_x+iv_x} = u_x-iv_x = \bar f_x$$
and similarly $\bar{f_y}=\bar f_y$. Using these facts,
$$\bar{\partial f} = \bar{\frac12(f_x-if_y)} = \frac12(\bar{f_x}+i\bar{f_y}) = \frac12(\bar f_x+i\bar f_y) = \bar\partial~\bar f$$
\sep



\tbf{P33.1.} \pf Fix a point $z_0$ and fix $r>0$. To show $f$ is holomorphic at $z_0$, we fix $z=x+iy\in D(z_0,r)$ and aim to show $f'(z)$ exists. The CR equations let us set
$$A := u_x(x,y) = v_y(x,y),
\quad B := v_x(x,y) = -u_y(x,y)$$
Since $u$ has continuous first partials, $u$ is differentiable at $(x,y)$, so that as $(h,k)\to(0,0)$,
$$u(x+h,y+k) = u(x,y) + hu_x(x,y) + ku_y(x,y) + o(\norm{(h,k)})
\imp u(x+h,y+k) - u(x,y) =  hA - kB + o(\norm{(h,k)})$$
and by similar arguments,
$$v(x+h,y+k) = v(x,y) + hv_x(x,y) + kv_y(x,y) + o(\norm{(h,k)})
\imp v(x+h,y+k) - v(x,y) = hB + kA + o(\norm{(h,k)})$$
Using these expressions,
$$\frac{f(x+iy+h+ik) - f(x+iy)}{h+ik}
= \frac{1}{h+ik}\sbr{u(x+h,y+k) - u(x,y) + iv(x+h,y+k) - iv(x,y)}$$
$$= \frac{1}{h+ik}\sbr{hA - kB + ihB + ikA + o(\norm{(h,k)})}
= \frac{1}{h+ik}\sbr{hA - kB + ihB + ikA} + \frac{o{\norm{(h,k)}}}{h+ik}$$
The second term vanishes as $(h,k)\to(0,0)$ since
$$\frac{o{\norm{(h,k)}}}{h+ik} \le \abs{\frac{o(\norm{(h,k)})}{h+ik}}
= \frac{\abs{o{\norm{(h,k)}}}}{\norm{(h,k)}}
\app[(h,k)][(0,0)] 0$$
The first term becomes
$$\frac{h-ik}{h^2+k^2}\sbr{hA - kB + ihB + ikA}
= \frac{1}{h^2+k^2}\sbr{h^2A + ih^2B - hkB + ihkA - ihkA + hkB + ik^2B + k^2A}$$
$$= \frac{1}{h^2+k^2}\sbr{A(h^2+k^2) + i(h^2+k^2)B}
= A + iB$$
From our calculations, we finally get
$$f'(z) = \lim_{w\to0}\frac{f(z+w)-f(z)}{w}
= \lim_{(h,k)\to(0,0)}\frac{f(x+iy+h+ik) - f(x+iy)}{h+ik}
= A + iB$$
\sep



\tbf{P36.3.} \pf Write $f=\frac PQ$ for polynomials $P$ and $Q$ with no common roots. Reorient our view of the complex plane so that $L$ coincides with the imaginary axis, with the roots $w_1,\dots,w_n$ of $P$ to the left and the roots $u_1,\dots,u_m$ of $Q$ to the right. With $a_n$ the lead coefficient of $P$,
$$P(z) = a_n\prod_{k=1}^n (z-w_k)
\imp P'(z) = a_n\sum_{k=1}^n\prod_{j=1\atop j\ne k}^n(z-w_j)
= a_n\sum_{k=1}^n\frac{(z-w_1)(z-w_n)}{z-w_k}
\imp \frac{P'(z)}{P(z)} = \sum_{k=1}^n\frac{1}{z-w_k}$$
For all $t\in\bR$, set $z=it$, then since $\Re w_k<0$,
$$\Re \frac{P'(z)}{P(z)} = \sum_{k=1}^n \Re\frac1{z-w_k}
= \sum_{k=1}^n \Re\frac{\bar{z-w_k}}{|z-w_k|^2}
= \sum_{k=1}^n\frac{\Re(z-w_k)}{|z-w_k|^2}
> 0$$
and by similar arguments and the fact $\Re u_k>0$,
$$\frac{Q'(z)}{Q(z)} = \sum_{k=1}^m\frac1{z-u_k}
\imp \Re\frac{Q'(z)}{Q(z)} = \sum_{k=1}^m\frac{\Re(z-u_k)}{|z-u_k|^2}
< 0$$
Putting the inequalities together,
$$\imp \Re \frac{P'(z)}{P(z)} \ne \Re\frac{Q'(z)}{Q(z)}
\imp \frac{P'(z)}{P(z)} \ne \frac{Q'(z)}{Q(z)}
\imp P'(z)Q(z)-P(z)Q'(z) \ne 0\quad (36.3.1)$$
By the quotient rule,
$$f'=\frac{P'Q-PQ'}{Q^2}$$
which along with (1) means $f'(it)\ne0$ for all $t\in\bR$.
\sep



\tbf{P40.2.} \pf Using the power series for $|z-w|<|w|$,
$$\Log z = \sum_{k\ge1}\frac{(-1)^{k+1}}{kw^k}(z-w)^k$$
set $z=2$ and $w=1$. Since $\Arg 2=0$, we have $\Log 2=\ln 2$.
$$\ln 2 = \sum_{k\ge1}\frac{(-1)^{k+1}}{k\cdot 1^k}1^k = \sum_{k\ge1}\frac{(-1)^{k+1}}{k}$$
\sep



\tbf{P41.6.} \pf Let $P:=(x_1,x_2,x_3)$ and $Q:=(y_1,y_2,y_3)$ be the points on the Riemann sphere whose stereographic projections are $z$ and $w$, respectively, i.e.
$$z = \frac{x_1+ix_2}{1-x_3},
\quad w = \frac{y_1+iy_2}{1-y_3}$$

($\imp$) If $P$ and $Q$ are diametrically opposite then $y_j=-x_j$ for $j=1,2,3$, so
$$z\bar w = \frac{(x_1+ix_2)(y_1-iy_2)}{(1-x_3)(1-y_3)}
= \frac{(x_1+ix_2)(-x_1+iy_2)}{(1-x_3)(1+x_3)}
= -\frac{x_1^2+x_2^2}{1-x_3^2}
= -\frac{1-x_3^2}{1-x_3^2}
= -1$$

($\impleft$) We first find
$$\bar ww = |w|^2
= \frac{y_1^2+y_2^2}{(1-y_3)^2}
= \frac{1-y_3^2}{(1-y_3)^2}
= \frac{1+y_3}{1-y_3}
\imp \frac{\bar ww}{w} = \frac{(1+y_3)(1-y_3)}{(1-y_3)(y_1+iy_2)}
= \frac{1+y_3}{y_1+iy_2}$$
Under the assumption $z\bar w=-1$,
$$-1 = z\bar w = \frac{z\bar ww}{w}
= \frac{(x_1+ix_2)(1+y_3)}{(1-x_3)(y_1+iy_2)}
\imp (x_1+ix_2)(1+y_3) = -(1-x_3)(y_1+iy_2)$$
Equating real and imaginary parts,
$$x_1(1+y_3) = -(1-x_3)y_1,~x_2(1+y_3) = -(1-x_3)y_2
\imp \frac{x_1}{y_1} = \frac{x_2}{y_2} = \frac{x_3-1}{y_3+1}$$
For convenience later, set $c:=\frac{x_1}{y_1}$, so that
$$x_1 = cy_1,~x_2 = cy_2,~x_3-1 = c(y_3+1)$$
Now consider the chord from $(0,0,1)$ to $P$. It has a direction vector $a:=(x_1,x_2,x_3-1)$. Similarly the chord from $(0,0,1)$ to $Q$ has a direction vector $b:=(y_1,y_2,y_3-1)$. We find
$$a\cdot b = x_1y_1 + x_2y_2 + (x_3-1)(y_3-1) = c(y_1^2+y_2^2+y_3^2-1) = c\cdot0 = 0$$
Using the fact $|a\cdot b|=\norm{a}\norm{b}\cos\th$, where $\th$ is the angle between the aforementioned chords, gives $\cos\th=0$ hence $\th=\frac\pi2$. From Euclidean geometry, the angle that $P$ and $Q$ subtend at the origin is $2\th=\pi$, i.e. $P$ and $Q$ are diametrically opposite.

	
\end{document}