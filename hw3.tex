\documentclass{article}
\usepackage{amsfonts, amsthm, amsmath, amssymb, mathtools, ulem, mathrsfs, physics, esint, siunitx, tikz-cd}
\usepackage{pdfpages, fullpage, color, microtype, cancel, textcomp, markdown, hyperref, graphicx}
\usepackage{enumitem}
\usepackage{algorithm}
\usepackage{algpseudocode}
\graphicspath{{./images/}}
\usepackage[english]{babel}
\usepackage[autostyle, english=american]{csquotes}
\MakeOuterQuote{"}
\usepackage{xparse}
\usepackage{tikz}

\usepackage{calligra}
\DeclareMathAlphabet{\mathcalligra}{T1}{calligra}{m}{n}
\DeclareFontShape{T1}{calligra}{m}{n}{<->s*[2.2]callig15}{}
\newcommand{\script}[1]{\ensuremath{\mathcalligra{#1}}}
\newcommand{\scr}{\script r}

% fonts
\def\mbb#1{\mathbb{#1}}
\def\mfk#1{\mathfrak{#1}}
\def\mbf#1{\mathbf{#1}}
\def\tbf#1{\textbf{#1}}

% common bold letters
\def\bP{\mbb{P}}
\def\bC{\mbb{C}}
\def\bH{\mbb{H}}
\def\bI{\mbb{I}}
\def\bR{\mbb{R}}
\def\bQ{\mbb{Q}}
\def\bZ{\mbb{Z}}
\def\bN{\mbb{N}}

% brackets
\newcommand{\br}[1]{\left(#1\right)}
\newcommand{\sbr}[1]{\left[#1\right]}
\newcommand{\brc}[1]{\left\{#1\right\}}
\newcommand{\lbr}[1]{\left\langle#1\right\rangle}

% vectors
\renewcommand{\i}{\hat{\imath}}
\renewcommand{\j}{\hat{\jmath}}
\renewcommand{\k}{\hat{k}}
\newcommand{\proj}[2]{\text{proj}_{#2}\br{#1}}
\newcommand{\m}[2][b]{\begin{#1matrix}#2\end{#1matrix}}
\newcommand{\arr}[3][\sbr]{#1{\begin{array}{#2}#3\end{array}}}

% misc
\NewDocumentCommand{\seq}{O{n} O{1} O{\infty} m}{\br{#4}_{{#1}={#2}}^{#3}}
\NewDocumentCommand{\app}{O{x} O{\infty}}{\xrightarrow{#1\to#2}}
\newcommand{\sm}{\setminus}
\newcommand{\sse}{\subseteq}
\renewcommand{\ss}{\subset}
\newcommand{\vn}{\varnothing}
\newcommand{\lc}{\epsilon_{ijk}}
\newcommand{\ep}{\epsilon}
\newcommand{\vp}{\varphi}
\renewcommand{\th}{\theta}
\newcommand{\cjg}[1]{\overline{#1}}
\newcommand{\inv}{^{-1}}
\DeclareMathOperator{\im}{im}
\DeclareMathOperator{\id}{id}
\newcommand{\ans}{\tbf{Ans. }}
\newcommand{\pf}{\tbf{Pf. }}
\newcommand{\imp}{\implies}
\newcommand{\impleft}{\reflectbox{$\implies$}}
\newcommand{\ck}{\frac1{4\pi\ep_0}}
\newcommand{\ckb}{4\pi\ep_0}
\newcommand{\sto}{\longrightarrow}
\DeclareMathOperator{\cl}{cl}
\DeclareMathOperator{\intt}{int}
\DeclareMathOperator{\bd}{bd}
\DeclareMathOperator{\Span}{span}
\newcommand{\floor}[1]{\left\lfloor#1\right\rfloor}
\newcommand{\ceil}[1]{\left\lceil#1\right\rceil}
\newcommand{\fxn}[5]{#1:\begin{array}{rcl}#2&\longrightarrow & #3\\[-0.5mm]#4&\longmapsto &#5\end{array}}
\newcommand{\sep}[1][.5cm]{\vspace{#1}}
\DeclareMathOperator{\card}{card}
\renewcommand{\ip}[2]{\lbr{#1,#2}}
\renewcommand{\bar}{\overline}
\DeclareMathOperator{\cis}{cis}
\DeclareMathOperator{\Arg}{Arg}
\DeclareMathOperator{\Ln}{Ln}

% title
\title{Complex Analysis HW 3}
\author{Ryan Chen}
%\date{\today}
\setlength{\parindent}{0pt}


\begin{document}
	
\maketitle



\tbf{P60.6.} \pf Fix $z\in R$ and write $f(z)=u+iv$.
$$|f(z) - 1| < 1 \imp (u-1)^2 \le (u-1)^2 + v^2 < 1
\imp |u - 1| < 1
\imp -1 < u - 1 < 1
\imp u > 0$$
so $f(R)$ is contained in the right half plane, on which $\Ln$ is analytic. With $f$ analytic on $R$, this means $\Ln f$ is analytic on $R$, and by the chain rule, $(\Ln f)'=\frac{f'}{f}$. Then $\gamma$ being closed gives us $\int_\gamma\frac{f'}{f}=0$.
\sep


\tbf{P60.7.} Since $Q_N$ is a square with side length $2N\pi$, the length of $Q_N$ is
$$L_N := \int_{Q_N}|dz| = 4 \cdot 2N\pi = 8N\pi$$
Set $f(z):=\frac1{z\cos z}$. We seek a bound on $f$ along $Q_N$. First examining the left and right sides of $Q_N$, write $z=\pm N\pi+ti,~-N\pi\le t\le N\pi$.
$$\cos z = \cos(\pm N\pi)\cos(it) - \sin(\pm N\pi)\sin(it) = (-1)^N\cosh t
\imp |\cos z| = \cosh t \ge 1
\imp \frac{1}{|\cos z|} \le 1$$
$$|z|^2 = (N\pi)^2 + t^2 \ge (N\pi)^2
\imp \frac{1}{|z|} \le \frac{1}{N\pi}$$
$$|f(z)| = \frac{1}{|z||\cos z|} \le \frac{1}{N\pi}$$
Then examining the top and bottom sides of $Q_N$, write $z=t\pm N\pi i,~-N\pi\le t\le N\pi$.
$$\cos z = \cos(t)\cos(\pm N\pi i) - \sin(t)\sin(\pm N\pi i) = \cosh N\pi\cos t \mp i\sinh N\pi \sin t $$
$$\imp |\cos z| = \cosh^2N\pi \cos^2t + \sinh^2N\pi \sin^2t
= (\sinh^2N\pi + 1)\cos^2t + \sinh^2N\pi \sin^2t$$
$$= \sinh^2N\pi(\cos^2t + \sin^2t) + \cos^2t
= \sinh^2N\pi+ \cos^2t
\ge \sinh^2N\pi
\ge \sinh^2\pi
\ge 1$$
$$\imp \frac{1}{|\cos z|} \le 1$$
$$|z|^2 = t^2 + (N\pi)^2 \ge (N\pi)^2 \imp \frac{1}{|z|} \le \frac{1}{N\pi}$$
$$|f(z)| = \frac{1}{|z||\cos z|} \le \frac{1}{N\pi}$$
Thus $|f|\le \frac{1}{N\pi}$ on $Q_N$. We then have a constant bound on the integral below.
$$\abs{\int_{Q_N}f(z)dz} \le \frac{1}{N\pi}L_N = \frac{8N\pi}{N\pi} = 8$$
\sep



\tbf{P66.1.} \pf With $f$ analytic hence holomorphic on $D$, for $z_0\in D$ there exists $r>0$ such that for $|z-z_0|<r$,
$$f(z) = \sum_{j\ge0}a_jz^j$$
Now set
$$F(z) := \sum_{j\ge0}\frac{1}{j+1}a_jz^{j+1}$$
so that $F'(z)=f(z)$ for $|z-z_0|<r$ (in particular $F'(z_0)=f(z_0)$), giving $F$ as holomorphic at $z_0$.
\sep



\tbf{P68.} First we find the Maclaurin series for $z\cosh z^2$. Using the Maclaurin series for exp,
$$e^z = \sum_{n\ge0}\frac{z^n}{n!} \imp e^{-z} = \sum{n\ge0}\frac{(-1)^nz^n}{n!}$$
Due to the $(-1)^n$ term, adding the two series will cancel the odd terms and double the even terms.
$$\cosh z = \frac12(e^z - e^{-z}) = \frac12\sum_{n\ge0}2\frac{z^{2n}}{(2n)!} = \sum_{n\ge0}\frac{z^{2n}}{(2n)!}$$
We then get the Maclaurin series for $z\cosh z^2$.
$$\imp \cosh z^2 = \sum_{n\ge0}\frac{z^{4n}}{(2n)!}
\imp \boxed{z\cosh z^2 = \sum_{n\ge0}\frac{z^{4n+1}}{(2n)!}}$$
Perform the ratio test for convergence.
$$\abs{\frac{z^{4n+5}}{(2n+2)!} \frac{(2n)!}{z^{4n+1}}} = \frac{|z|^4}{(2n+2)(2n+1)} \app[n][\infty] 0$$
The series converges absolutely for all $z$.\\

Using the formula for the geometric series,
$$\frac{1}{1-z} = \sum_{n\ge0}z^n
\imp \frac{1}{z^4+9} = \frac19 \frac{1}{1-(-z^4/9)}
= \frac19 \sum_{n\ge0}\br{-\frac{z^4}{9}}^n
= \frac19 \sum_{n\ge0}\frac{(-1)^nz^{4n}}{9^n}$$
$$\imp \boxed{\frac{z}{z^4+9} = \frac19 \sum_{n\ge0}\frac{(-1)^nz^{4n+1}}{9^n}}$$
Perform the ratio test. The series converges absolutely iff
$$\abs{\frac{(-1)^{n+1}z^{4n+5}}{9^{n+1}} \frac{9^n}{(-1)^nz^{4n+1}}} = \frac{|z|^4}{9} < 1
\iff |z|^4 < 9
\iff |z| < \sqrt3$$
\sep



\tbf{P71.3.} \pf For all $z$, the sequence $z+n$ diverges to $\infty$ as $n\to\infty$, so $f_n$ has no piecewise limit, hence no uniform limit, on any (nonempty) subset of $\bC$. Thus $f_n$ does not converge normally to any function $\bC\to\bC$.\\

Now fix $z=x+yi\in\bC$ and consider the sequence $z+n$ on the Riemann sphere with the spherical metric $\chi$. It converges to $\infty$ since
$$\chi(z+n,\infty) = \frac{2}{[1+|z+n|^2]^{1/2}}
= \frac{2}{[1+(x+n)^2+y^2]^{1/2}}
= \frac{2}{[n^2+2xn+x^2+y^2+1]^{1/2}}
\le \frac{2}{[n^2+2xn]^{1/2}} \cdot \frac{1/n}{[1/n^2]^{1/2}}$$
$$= \frac{2/n}{[1+2x/n]^{1/2}}
\app[n][\infty]
\frac{0}{[1+0]^{1/2}}
= 0$$

\end{document}