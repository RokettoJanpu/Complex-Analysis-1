\documentclass{article}
\usepackage{amsfonts, amsthm, amsmath, amssymb, mathtools, ulem, mathrsfs, physics, esint, siunitx, tikz-cd}
\usepackage{pdfpages, fullpage, color, microtype, cancel, textcomp, markdown, hyperref, graphicx}
\usepackage{enumitem}
\usepackage{algorithm}
\usepackage{algpseudocode}
\graphicspath{{./images/}}
\usepackage[english]{babel}
\usepackage[autostyle, english=american]{csquotes}
\MakeOuterQuote{"}
\usepackage{xparse}
\usepackage{tikz}

\usepackage{calligra}
\DeclareMathAlphabet{\mathcalligra}{T1}{calligra}{m}{n}
\DeclareFontShape{T1}{calligra}{m}{n}{<->s*[2.2]callig15}{}
\newcommand{\script}[1]{\ensuremath{\mathcalligra{#1}}}
\newcommand{\scr}{\script r}

% fonts
\def\mbb#1{\mathbb{#1}}
\def\mfk#1{\mathfrak{#1}}
\def\mbf#1{\mathbf{#1}}
\def\tbf#1{\textbf{#1}}

% common bold letters
\def\bP{\mbb{P}}
\def\bC{\mbb{C}}
\def\bH{\mbb{H}}
\def\bI{\mbb{I}}
\def\bR{\mbb{R}}
\def\bQ{\mbb{Q}}
\def\bZ{\mbb{Z}}
\def\bN{\mbb{N}}

% brackets
\newcommand{\br}[1]{\left(#1\right)}
\newcommand{\sbr}[1]{\left[#1\right]}
\newcommand{\brc}[1]{\left\{#1\right\}}
\newcommand{\lbr}[1]{\left\langle#1\right\rangle}

% vectors
\renewcommand{\i}{\hat{\imath}}
\renewcommand{\j}{\hat{\jmath}}
\renewcommand{\k}{\hat{k}}
\newcommand{\proj}[2]{\text{proj}_{#2}\br{#1}}
\newcommand{\m}[2][b]{\begin{#1matrix}#2\end{#1matrix}}
\newcommand{\arr}[3][\sbr]{#1{\begin{array}{#2}#3\end{array}}}

% misc
\NewDocumentCommand{\seq}{O{n} O{1} O{\infty} m}{\br{#4}_{{#1}={#2}}^{#3}}
\NewDocumentCommand{\app}{O{x} O{\infty}}{\xrightarrow{#1\to#2}}
\newcommand{\sm}{\setminus}
\newcommand{\sse}{\subseteq}
\renewcommand{\ss}{\subset}
\newcommand{\vn}{\varnothing}
\newcommand{\lc}{\epsilon_{ijk}}
\newcommand{\ep}{\epsilon}
\newcommand{\vp}{\varphi}
\renewcommand{\th}{\theta}
\newcommand{\cjg}[1]{\overline{#1}}
\newcommand{\inv}{^{-1}}
\DeclareMathOperator{\im}{im}
\DeclareMathOperator{\id}{id}
\newcommand{\ans}{\tbf{Ans. }}
\newcommand{\pf}{\tbf{Pf. }}
\newcommand{\imp}{\implies}
\newcommand{\impleft}{\reflectbox{$\implies$}}
\newcommand{\ck}{\frac1{4\pi\ep_0}}
\newcommand{\ckb}{4\pi\ep_0}
\newcommand{\sto}{\longrightarrow}
\DeclareMathOperator{\cl}{cl}
\DeclareMathOperator{\intt}{int}
\DeclareMathOperator{\bd}{bd}
\DeclareMathOperator{\Span}{span}
\newcommand{\floor}[1]{\left\lfloor#1\right\rfloor}
\newcommand{\ceil}[1]{\left\lceil#1\right\rceil}
\newcommand{\fxn}[5]{#1:\begin{array}{rcl}#2&\longrightarrow & #3\\[-0.5mm]#4&\longmapsto &#5\end{array}}
\newcommand{\sep}[1][.5cm]{\vspace{#1}}
\DeclareMathOperator{\card}{card}
\renewcommand{\ip}[2]{\lbr{#1,#2}}
\renewcommand{\bar}{\overline}
\DeclareMathOperator{\cis}{cis}
\DeclareMathOperator{\Arg}{Arg}

% title
\title{Complex Analysis Quiz 3}
\author{Ryan Chen}
%\date{\today}
\setlength{\parindent}{0pt}


\begin{document}
	
\maketitle



\begin{enumerate}
	
\item Write
$$h(\alpha) = \int_0^\infty f(x)dx,
\quad f(z) := \frac{1}{z^\alpha+1}$$
The simple poles $z_k$ of $f$ are
$$z^\alpha = -1 = e^{i(2k+1)\pi}
\imp z_k = e^{i(2k+1)\pi/\alpha}$$
Consider the contour
$$\Gamma := [0,R] + C_R - \gamma_R,
\quad C_R: z=Re^{it},~0\le t\le\frac{2\pi}{\alpha},
\quad \gamma_R: z=te^{i2\pi/\alpha},~0\le t\le R$$
so that
$$\int_\Gamma f(z)dz = \int_{[0,R]}f(z)dz + \int_{C_R}f(z)dz - \int_{\gamma_R}f(z)dz$$
and the only pole of $f$ that lies inside $\Gamma$ is $z_0=e^{i\pi/\alpha}$. We find
$$\Res(f,z_0) = \lim_{z\to z_0}\frac{z-z_0}{z^\alpha+1}$$
Using L'Hopital's rule,
$$\Res(f,z_0) = \lim_{z\to z_0}\frac{1}{\alpha z^{\alpha-1}}
= \frac{1}{\alpha z_0^{\alpha-1}}
= \frac{z_0}{\alpha z_0^\alpha}
= -\frac{z_0}{\alpha}$$
Thus by the residue theorem,
$$\int_\Gamma f(z)dz = 2\pi i\Res(f,z_0)
= -2\pi i\frac{z_0}{\alpha}
= -2\pi i\frac{e^{i\pi/\alpha}}{\alpha}$$

On $C_R$,
$$|z| = R
\imp |f(z)| \le \frac{1}{||z|^\alpha-1|}
= \frac{1}{R^\alpha-1}
= O\br{\frac{1}{R^\alpha}}$$
so that, using the fact $\alpha>1$,
$$\abs{\int_{C_R}f(z)dz} = O\br{\frac{1}{R^{\alpha-1}}}
\app[R][\infty] 0$$
Then we compute
$$\int_{\gamma_R}f(z)dz = \int_0^R\frac{1}{t^\alpha e^{i2\pi}+1}e^{i2\pi/\alpha}dt
= e^{i2\pi/\alpha}\int_0^R f(t)dt$$

Putting together the computations, as $R\to\infty$,
$$-2\pi i\frac{e^{i\pi/\alpha}}{\alpha} = \int_0^\infty f(x)dx - e^{i2\pi/\alpha}\int_0^R f(x)dx
= \sbr{1 - e^{i2\pi/\alpha}}\int_0^\infty f(x)dx$$
Multiply by $e^{-i\pi/\alpha}$.
$$\frac{-2\pi i}{\alpha} = \sbr{e^{-i\pi/\alpha} - e^{i\pi/\alpha}}\int_0^\infty f(x)dx
= -2i\sin\frac\pi\alpha\int_0^\infty f(x)dx$$
$$\imp h(\alpha) = \int_0^\infty f(x)dx = \frac{\pi/\alpha}{\sin\pi/\alpha}$$


\item From the last part, the expression for $h(\alpha)$ is valid for
$$\alpha\in\bC-\sbr{\brc{\frac1k:k\in\bZ}\cup\brc0}$$

\end{enumerate}


	
\end{document}