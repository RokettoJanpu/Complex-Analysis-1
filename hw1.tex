\documentclass{article}
\usepackage{amsfonts, amsthm, amsmath, amssymb, mathtools, ulem, mathrsfs, physics, esint, siunitx, tikz-cd}
\usepackage{pdfpages, fullpage, color, microtype, cancel, textcomp, markdown, hyperref, graphicx}
\usepackage{enumitem}
\usepackage{algorithm}
\usepackage{algpseudocode}
\graphicspath{{./images/}}
\usepackage[english]{babel}
\usepackage[autostyle, english=american]{csquotes}
\MakeOuterQuote{"}
\usepackage{xparse}
\usepackage{tikz}

\usepackage{calligra}
\DeclareMathAlphabet{\mathcalligra}{T1}{calligra}{m}{n}
\DeclareFontShape{T1}{calligra}{m}{n}{<->s*[2.2]callig15}{}
\newcommand{\script}[1]{\ensuremath{\mathcalligra{#1}}}
\newcommand{\scr}{\script r}

% fonts
\def\mbb#1{\mathbb{#1}}
\def\mfk#1{\mathfrak{#1}}
\def\mbf#1{\mathbf{#1}}
\def\tbf#1{\textbf{#1}}

% common bold letters
\def\bP{\mbb{P}}
\def\bC{\mbb{C}}
\def\bH{\mbb{H}}
\def\bI{\mbb{I}}
\def\bR{\mbb{R}}
\def\bQ{\mbb{Q}}
\def\bZ{\mbb{Z}}
\def\bN{\mbb{N}}

% brackets
\newcommand{\br}[1]{\left(#1\right)}
\newcommand{\sbr}[1]{\left[#1\right]}
\newcommand{\brc}[1]{\left\{#1\right\}}
\newcommand{\lbr}[1]{\left\langle#1\right\rangle}

% vectors
\renewcommand{\i}{\hat{\imath}}
\renewcommand{\j}{\hat{\jmath}}
\renewcommand{\k}{\hat{k}}
\newcommand{\proj}[2]{\text{proj}_{#2}\br{#1}}
\newcommand{\m}[2][b]{\begin{#1matrix}#2\end{#1matrix}}
\newcommand{\arr}[3][\sbr]{#1{\begin{array}{#2}#3\end{array}}}

% misc
\NewDocumentCommand{\seq}{O{n} O{1} O{\infty} m}{\br{#4}_{{#1}={#2}}^{#3}}
\NewDocumentCommand{\app}{O{x} O{\infty}}{\xrightarrow{#1\to#2}}
\newcommand{\sm}{\setminus}
\newcommand{\sse}{\subseteq}
\renewcommand{\ss}{\subset}
\newcommand{\vn}{\varnothing}
\newcommand{\lc}{\epsilon_{ijk}}
\newcommand{\ep}{\epsilon}
\newcommand{\vp}{\varphi}
\renewcommand{\th}{\theta}
\newcommand{\cjg}[1]{\overline{#1}}
\newcommand{\inv}{^{-1}}
\DeclareMathOperator{\im}{im}
\DeclareMathOperator{\id}{id}
\newcommand{\ans}{\tbf{Ans. }}
\newcommand{\pf}{\tbf{Pf. }}
\newcommand{\imp}{\implies}
\newcommand{\impleft}{\reflectbox{$\implies$}}
\newcommand{\ck}{\frac1{4\pi\ep_0}}
\newcommand{\ckb}{4\pi\ep_0}
\newcommand{\sto}{\longrightarrow}
\DeclareMathOperator{\cl}{cl}
\DeclareMathOperator{\intt}{int}
\DeclareMathOperator{\bd}{bd}
\DeclareMathOperator{\Span}{span}
\newcommand{\floor}[1]{\left\lfloor#1\right\rfloor}
\newcommand{\ceil}[1]{\left\lceil#1\right\rceil}
\newcommand{\fxn}[5]{#1:\begin{array}{rcl}#2&\longrightarrow & #3\\[-0.5mm]#4&\longmapsto &#5\end{array}}
\newcommand{\sep}[1][.5cm]{\vspace{#1}}
\DeclareMathOperator{\card}{card}
\renewcommand{\ip}[2]{\lbr{#1,#2}}
\renewcommand{\bar}{\overline}
\DeclareMathOperator{\cis}{cis}
\DeclareMathOperator{\Arg}{Arg}

% title
\title{Complex Analysis HW 1}
\author{Ryan Chen}
\date{}
\setlength{\parindent}{0pt}


\begin{document}
	
\maketitle



\tbf{Note.} Define $\cis\theta:=\cos\theta+i\sin\theta$.
\sep

	

\tbf{9.2.} \pf Write $z=x+yi$ and $w=u+vi$.

\begin{enumerate}
	
\item
$$\bar{z+w} = \bar{(x+u)+(y+v)i} = (x+u) - (y+v)i = (x-yi) + (u-vi) = \bar z + \bar w$$

\item
$$\bar{zw} = \bar{(xu-yv)+(xv+yu)i} = (xu-yv) - (xv+yu)i = xu - xvi - yui - yv = (x-yi)(u-vi) = \bar z\bar w$$

\end{enumerate}
\sep



\tbf{11.2.} \ans Set $r:=|z|,~s:=|w|,~\theta:=\Arg z,~\vp:=\Arg w$. Then $zw=rs\cis(\theta+\vp)$, so that
$$\Arg(zw) = \theta+\vp
\iff -\pi < \theta+\vp \le \pi
\iff -\pi-\vp< \theta \le \pi-\vp$$
\sep



\tbf{13.2.} \ans
$$(z-1)^4 = z^4
\iff z^4-(z-1)^4=0
\iff [z^2-(z-1)^2][z^2+(z-1)^2]=0
\iff z^2-(z-1)^2=0 \text{ or } z^2+(z-1)^2=0$$
In the first above case,
$$0 = z^2-(z^2-2z+1) = 2z-1
\iff z=\frac12$$
In the second above case,
$$0 = z^2+z^2-2z+1 = 2z^2-2z+1
\iff z = \frac14\sbr{2\pm\sqrt{4-4(2)(1)}}
= \frac14\sbr{2\pm\sqrt{-4}}
= \frac14\sbr{2\pm 2i}
= \frac12\sbr{1\pm i}$$
Thus the solutions are $z=\frac12,\frac12\sbr{1\pm i}$.
\sep



\tbf{15.2.} \pf Since $|r\cis\th|=|r||\cis\th|=|r|<1$, the following geometric series converges.
$$\sum_{n\ge0}(r\cis\th)^n = \frac{1}{1-r\cis\th}
= \frac{1-r\cos\th+ir\sin\th}{(1-r\cos\th-ir\sin\th)(1-r\cos\th+ir\sin\th)}
= \frac{1-r\cos\th+ir\sin\th}{(1-r\cos\th)^2+r^2\sin^2\th}$$
$$= \frac{1-r\cos\th+ir\sin\th}{1-2r\cos\th+r^2\cos^2\th+r^2\sin^2\th}
= \frac{1-r\cos\th+ir\sin\th}{1+r^2-2r\cos\th}$$
Now by de Moivre's formula, for all $n\ge0$,
$$(r\cis\th)^n = r^n\cis^n\th = r^n\cis n\th = r^n\cos n\th+ir^n\sin n\th$$
so that
$$\sum_{n\ge0}(r\cis\th)^n = \sum_{n\ge0}r^n\cos n\th + i\sum_{n\ge0}r^n\sin n\th$$
Using the above calculation,
$$\sum_{n\ge0}r^n\cos n\th + i\sum_{n\ge0}r^n\sin n\th = \frac{1-r\cos\th+ir\sin\th}{1+r^2-2r\cos\th}$$
Equating real parts,
$$\sum_{n\ge0}r^n\cos n\th = \frac{1-r\cos\th}{1+r^2-2r\cos\th}$$
\sep



\tbf{26.2.} \pf Euler's formula gives
$$\exp[i(a+b)] = \cos(a+b) + i\sin(a+b)$$
The addition formula for exp gives
$$\exp[i(a+b)] = \exp[ia+ib]
= \exp(ia)\exp(ib)$$
$$= (\cos a+i\sin a)(\cos b+i\sin b)
= [\cos a\cos b - \sin a\sin b] + i[\cos a\sin b + \sin a\cos b]$$
Equating real and imaginary parts,
$$\cos(a+b) = \cos a\cos b - \sin a\sin b$$
$$\sin(a+b) = \cos a\sin b + \sin a\cos b$$
	
\end{document}