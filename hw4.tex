\documentclass{article}
\usepackage{amsfonts, amsthm, amsmath, amssymb, mathtools, ulem, mathrsfs, physics, esint, siunitx, tikz-cd}
\usepackage{pdfpages, fullpage, color, microtype, cancel, textcomp, markdown, hyperref, graphicx}
\usepackage{enumitem}
\usepackage{algorithm}
\usepackage{algpseudocode}
\graphicspath{{./images/}}
\usepackage[english]{babel}
\usepackage[autostyle, english=american]{csquotes}
\MakeOuterQuote{"}
\usepackage{xparse}
\usepackage{tikz}

\usepackage{calligra}
\DeclareMathAlphabet{\mathcalligra}{T1}{calligra}{m}{n}
\DeclareFontShape{T1}{calligra}{m}{n}{<->s*[2.2]callig15}{}
\newcommand{\script}[1]{\ensuremath{\mathcalligra{#1}}}
\newcommand{\scr}{\script r}

% fonts
\def\mbb#1{\mathbb{#1}}
\def\mfk#1{\mathfrak{#1}}
\def\mbf#1{\mathbf{#1}}
\def\tbf#1{\textbf{#1}}

% common bold letters
\def\bP{\mbb{P}}
\def\bC{\mbb{C}}
\def\bH{\mbb{H}}
\def\bI{\mbb{I}}
\def\bR{\mbb{R}}
\def\bQ{\mbb{Q}}
\def\bZ{\mbb{Z}}
\def\bN{\mbb{N}}

% brackets
\newcommand{\br}[1]{\left(#1\right)}
\newcommand{\sbr}[1]{\left[#1\right]}
\newcommand{\brc}[1]{\left\{#1\right\}}
\newcommand{\lbr}[1]{\left\langle#1\right\rangle}

% vectors
\renewcommand{\i}{\hat{\imath}}
\renewcommand{\j}{\hat{\jmath}}
\renewcommand{\k}{\hat{k}}
\newcommand{\proj}[2]{\text{proj}_{#2}\br{#1}}
\newcommand{\m}[2][b]{\begin{#1matrix}#2\end{#1matrix}}
\newcommand{\arr}[3][\sbr]{#1{\begin{array}{#2}#3\end{array}}}

% misc
\NewDocumentCommand{\seq}{O{n} O{1} O{\infty} m}{\br{#4}_{{#1}={#2}}^{#3}}
\NewDocumentCommand{\app}{O{x} O{\infty}}{\xrightarrow{#1\to#2}}
\newcommand{\sm}{\setminus}
\newcommand{\sse}{\subseteq}
\renewcommand{\ss}{\subset}
\newcommand{\vn}{\varnothing}
\newcommand{\lc}{\epsilon_{ijk}}
\newcommand{\ep}{\epsilon}
\newcommand{\vp}{\varphi}
\renewcommand{\th}{\theta}
\newcommand{\cjg}[1]{\overline{#1}}
\newcommand{\inv}{^{-1}}
\DeclareMathOperator{\im}{im}
\DeclareMathOperator{\id}{id}
\newcommand{\ans}{\tbf{Ans. }}
\newcommand{\pf}{\tbf{Pf. }}
\newcommand{\imp}{\implies}
\newcommand{\impleft}{\reflectbox{$\implies$}}
\newcommand{\ck}{\frac1{4\pi\ep_0}}
\newcommand{\ckb}{4\pi\ep_0}
\newcommand{\sto}{\longrightarrow}
\DeclareMathOperator{\cl}{cl}
\DeclareMathOperator{\intt}{int}
\DeclareMathOperator{\bd}{bd}
\DeclareMathOperator{\Span}{span}
\newcommand{\floor}[1]{\left\lfloor#1\right\rfloor}
\newcommand{\ceil}[1]{\left\lceil#1\right\rceil}
\newcommand{\fxn}[5]{#1:\begin{array}{rcl}#2&\longrightarrow & #3\\[-0.5mm]#4&\longmapsto &#5\end{array}}
\newcommand{\sep}[1][.5cm]{\vspace{#1}}
\DeclareMathOperator{\card}{card}
\renewcommand{\ip}[2]{\lbr{#1,#2}}
\renewcommand{\bar}{\overline}
\DeclareMathOperator{\cis}{cis}
\DeclareMathOperator{\Arg}{Arg}
\newcommand{\ptl}{\partial}

% title
\title{Complex Analysis HW 4}
\author{Ryan Chen}
%\date{\today}
\setlength{\parindent}{0pt}


\begin{document}

\maketitle



\tbf{Note.} Write $\cis t:=\cos t+i\sin t$.
\sep



\tbf{P81.3.} Set
$$g(z) := \frac{z^3}{27}f(z)$$
so that
$$|g(z)| = \frac{|z|^3}{27}|f(z)|$$
Then on $|z|=3$,
$$|g(z)| \le \frac{3^3}{27}\cdot 1 = 1$$
and on $|z|=1$,
$$|g(z)| \le \frac{1^3}{27}\cdot 27 = 1$$
Thus by the maximum principle, on $1\le |z|\le 3$,
$$|g(z)|\le 1 \imp |f(z)| \le \frac{27}{|z|^3}$$



\tbf{P94.2} From $u(z)=z^2+\bar z^2$, set
$$g(z) := \partial u(z) = 2z$$
$$G(z) := \int g(z)dz = z^2$$
$$f(z) := 2G(z) = 2z^2$$
We see $f$ is entire and check that
$$\frac12\br{f(z)+\bar{f(z)}} = \frac12\br{2z^2+2\bar z^2} = z^2 + \bar z^2 = u(z)$$
\sep



\tbf{P94.4} \pf With $u$ harmonic, there exists an entire function $f$ such that $\Re f=u$. Then $e^f$ is entire and
$$|e^f| = e^u < e^c$$
so by Liouville's theorem $e^f$ is constant, hence $f$ is constant. In turn, $u$ is constant.
\sep



\tbf{P96.4} Set $u(x,y):=y$ and $v(x,y):=\cosh\pi x\sin\pi y$.
$$v_{xx} = \pi^2\cosh\pi x\sin\pi y,
\quad v_{yy} = -\pi^2\cosh\pi x\sin\pi y
\imp v_{xx} + v_{yy} = 0$$
Thus $v$ is harmonic. Also, $v\eval_{y=0}=v\eval_{y=1}=0$. Then the function $w:=u+v$ is a sum of harmonic functions hence harmonic, and $w$ satisfies the boundary conditions $w\eval_{y=0}=0$ and $w\eval_{y=1}=1$.
\sep



\tbf{P107} \pf Fix $z$ in the unbounded component of $\alpha^c$. The winding number is
$$n(z,\alpha) = \frac{1}{2\pi i}\int_\alpha \frac{dw}{w-z}$$
Since $\alpha$ is bounded and closed, $\frac{1}{|w-z|}$ is continuous in $w$ on $\alpha$, so it attains a maximum at some $w_0\in\alpha$.
$$|n(z,\alpha)| \le \frac{1}{2\pi}\abs{\int_\alpha\frac{dw}{w-z}}
\le \frac{\mathrm{length}(\alpha)}{2\pi|w_0-z|}
\app[z][\infty] 0$$
so that $\lim_{z\to\infty}|n(z,\alpha)|=0$. Thus $|n(z,\alpha)|<1$ for large enough $|z|$. Now $n(z,\alpha)\in\bZ$, so $|n(z,\alpha)|=0$ hence $n(z,\alpha)=0$ for large enough $|z|$. Also $n(z,\alpha)$ must be constant on components of $\alpha^c$, thus $n(z,\alpha)=0$ on the unbounded component of $\alpha^c$.
\sep



\tbf{P114.1} The Laurent expansions are:

\begin{enumerate}[label=(\roman*)]
	
\item For $0<|z|<2$,
$$\frac{1}{z(z^2+4)} = \frac{1}{4z(1-(-z^2/4))}
= \frac{1}{4z}\sum_{k\ge0}\frac{(-1)^kz^{2k}}{4^k}
= \sum_{k\ge0}\frac{(-1)^kz^{2k-1}}{4^{k+1}}$$

\item For $|z|>2$,
$$\frac{1}{z(z^2+4)} = \frac{1}{z^3(1-(-4/z^2))}
= \frac{1}{z^3}\sum_{k\ge0}\frac{(-1)^k4^k}{z^{2k}}
= \sum_{k\ge0}\frac{(-1)^k4^k}{z^{2k+3}}$$

\item For $0<|z+2i|<2$,
$$\frac{1}{z(z^2+4)} = \frac{1}{z(z-2i)(z+2i)}
= \frac{1}{z(z-2i)}\sum_{k\ge0}(z+2i)^k$$

\end{enumerate}
\sep



\tbf{P119.1}

\begin{enumerate}[label=(\roman*)]
	
\item The function
$$f(z) = \frac{1}{(z-4)(z^3-1)}$$
has singularities at $z=4$ and $z=\cis\frac{2\pi k}{3},~0\le k\le 2$, i.e. $z=4,1,-\frac12+\frac{\sqrt3}{2}i,-\frac12+\frac{\sqrt3}{2}i$. Considering the circle $\alpha:|z-2|=\frac52$ has leftmost endpoint $-\frac12$, the singularities that lie inside $\alpha$ are $z=4,1$. We now find
$$(z-4)f(z) = \frac{1}{z^3-1}
\app[z][4] \frac{1}{64-1}
= \frac{1}{63}$$
so that $\Res(f,4)=\frac{1}{63}$, and
$$(z-1)f(z) = \frac{1}{(z-4)(z^2+z+1)}
\app[z][1] \frac{1}{-3(1+1+1)}
= -\frac19$$
so that $\Res(f,1)=-\frac19$. Thus
$$\int_\alpha f(z)dz = 2\pi i[\Res(f,4) + \Res(f,1)]
= 2\pi i\sbr{\frac{1}{63} - \frac19}
= 2\pi i\sbr{-\frac{6}{63}}
= -\frac{4\pi i}{21}$$


\item The function
$$g(z) = \frac{3z^2+2}{(z-1)(z^2+9)}$$
has singularities at $z=1,\pm 3i$ which all lie within the circle $\beta:|z|=5$. We find
$$(z-1)g(z) = \frac{3z^2+2}{z^2+9}
\app[z][1] \frac{3+2}{1+9}
= \frac{5}{10} 
= \frac12$$
so that $\Res(g,1)=\frac12$, and
$$(z\mp 3i)g(z) = \frac{3z^2+2}{(z-1)(z\pm 3i)}
\app[z][\pm 3i] \frac{3(-9)+2}{(\pm 3i-1)(\pm 6i)}
= \frac{-25}{-18\mp 6i}$$
$$= \frac{25}{18\pm 6i}\frac{18\mp 6i}{18\mp 6i}
= \frac{450\mp 150i}{324+36}
= \frac{450\mp 150i}{360}
= \frac{15\mp 5i}{12}$$
so that $\Res(g,\pm 3i)=\frac{15\mp 5i}{12}$. Thus
$$\int_\beta g(z)dz = 2\pi i\sbr{\Res(g,1) + \Res(g,3i) + \Res(g,-3i)}
= 2\pi i\sbr{\frac12 + \frac{15-5i}{12} + \frac{15+5i}{12}}$$
$$= 2\pi i\sbr{\frac12 + \frac{30}{12}}
= 2\pi i\sbr{\frac12 + \frac52}
= 6\pi i$$
	
\end{enumerate}
\sep



\tbf{P119.2} 

\begin{enumerate}[label=(\roman*)]
	
\item The function
$$f(z) = \frac{1}{\cosh2z}$$
has singularities at (write $w=2z=x+iy$)
$$\cosh2z = 0
\imp \cosh x\cos y+i\sinh x\sin y = 0
\imp \cosh x\cos y = 0,~\sinh x\sin y = 0$$
The first equation gives
$$\cos y = 0
\imp y = \frac\pi2+ \pi n,~n\in\bZ$$
The second equation gives $x=0$ or $y=\pi n$, so it must be the case $x=0$ and $y=\frac\pi2+\pi n$.
$$w = i\br{\frac\pi2+\pi n}
\imp z = \frac w2 = \br{\frac\pi4+\frac\pi2n}i$$
The only singularities that lie within the circle $\gamma:|z|=1$ are for $n=0,-1$, i.e. $z=\pm\frac\pi4i$. Label the points as $z_1$ and $z_2$.
$$\Res(f,z_j) = \lim_{z\to z_j}(z-z_j)f(z)
= \lim_{z\to z_j}\frac{z-z_j}{\cosh2z}$$
Using L'Hoptital's rule,
$$\Res(f,z_j) = \lim_{z\to z_j}(z-z_j)f(z)\frac{1}{2\sinh2z}
= \frac{1}{2\sinh2z_j}
= \frac{1}{2\sinh(\pm\frac\pi2i)}
= \frac{1}{2i\sin(\pm\frac\pi2)}
= -\frac{i}{2(\pm 1)}
= \mp\frac i2$$
Thus
$$\int_\gamma f(z)dz = 2\pi i\sbr{\Res(f,\frac\pi4i) + \Res(f,-\frac\pi4i)}
= 2\pi i\sbr{-\frac i2+\frac i2}
= 0$$


\item The function
$$g(z) = \cot z = \frac{\cos z}{\sin z}$$
has singularities at
$$\sin z = 0
\imp z = k\pi,~k\in\bZ$$
The only singularity that lies within the circle $\gamma:|z|=1$ is for $k=0$, i.e. $z=0$.
$$\Res(g,0) = \lim_{z\to 0}zg(z)
= \lim_{z\to 0}\frac{z\cos z}{\sin z}$$
Using L'Hopital's rule,
$$\Res(g,0) = \lim_{z\to 0}\frac{\cos z-z\sin z}{\cos z}
= \frac{1 - 0}{1}
= 1$$
Thus
$$\int_\gamma g(z)dz = 2\pi i\Res(g,0) = 2\pi i$$

\end{enumerate}
\sep



\tbf{P127.4} \pf Set
$$I := \int_0^\infty f(x)dx,
\quad f(z) := e^{iz^2}$$
Fix large $R>0$ and define the contours
$$\alpha := [0,R] + C_R - \gamma_R,
\quad C_R: z = Re^{it},~ 0\le t\le \frac\pi4,
\quad \gamma_R: z = te^{i\pi/4},~ 0\le t\le R$$
Since $f$ is analytic on the region enclosed by $\alpha$,
$$0 = \int_\alpha f(z)dz = \int_{[0,R]}f(z)dz + \int_{C_R}f(z)dz - \int_{\gamma_R}f(z)dz \quad (127.4.1)$$

We compute the third integral. On $\gamma_R$,
$$z = te^{i\pi/4},~0\le t\le R
\imp z^2 = t^2e^{i\pi/2} = it^2
\imp f(z) = e^{-it^2}$$
Using the Gaussian integral,
$$\int_{\gamma_R}f(z)dz = e^{i\pi/4}\int_0^R e^{-t^2}dt
\app[R][\infty] e^{i\pi/4}\int_0^\infty e^{-t^2}dt
= e^{i\pi/4}\frac{\sqrt\pi}{2}$$


We compute the second integral. On $C_R$,
$$z = Re^{it},~0\le t\le\frac\pi4,
\imp z^2 = R^2e^{i2t} = R^2\cos2t + iR^2\sin2t
\imp iz^2 = iR^2\cos2t - R^2\sin2t
\imp |f(z)| = e^{-R^2\sin2t}$$
Now using the fact $\sin t\ge\frac{2t}{\pi}$ for $0\le t\le\frac\pi2$, we have $\sin2t\ge\frac{4t}{\pi}$ for $0\le t\le\frac\pi4$, hence $|f(z)| \le e^{-R^24t/\pi}$.
$$\abs{\int_{C_R}f(z)dz} \le \int_{C_R}|f(z)||dz|
\le \int_0^{\pi/4}e^{-R^24t/\pi}Rdt
= -\frac{R\pi}{4R^2}e^{-4R^2t/\pi}\eval_0^{\pi/4}
= -\frac{\pi}{4R}(e^{-R^2}-1)
\app[R][\infty] 0(0-1)
= 0$$

Putting together the calculations, as we take $R\to\infty$ in equation (127.4.1),
$$0 = I - e^{i\pi/4}\frac{\sqrt\pi}{2}
\imp I = e^{i\pi/4}\frac{\sqrt\pi}{2} = \br{\frac{1}{\sqrt2} + i\frac{1}{\sqrt2}}\frac{\sqrt\pi}{2}
\imp \int_0^\infty \cos(x^2)dx = \im I = \frac{\sqrt\pi}{2\sqrt2}$$
\sep



\tbf{P131.1} Set $D:=\brc{|z|<1}$, so that $\ptl D=\brc{|z|=1}$.

\begin{enumerate}[label=(\roman*)]
	
\item Write
$$h(z) = f(z) + g(z),
\quad f(z) = 5z^4,
\quad g(z) = z^2 + 2z - i$$
On $\ptl D$,
$$|f(z)| = 5|z|^4 = 5,
\quad |g(z)| \le |z|^2 + 2|z| + |i| = 4
\imp |f(z)|>|g(z)|$$
In $D$, $f$ has a root at $z=0$ of order 4, so by Rouche's theorem $h$ has 4 roots in $D$.


\item Write
$$h(z) = g(z) + f(z),
\quad g(z) = z^3,
\quad f(z) = 8z^2 - 8z + 2$$
On $\ptl D$, using the reverse triangle inequality,
$$|f(z)| \ge \abs{8|z|^2 - 8|z| - |2|} = |8-8-2| = 2,
\quad |g(z)| = |z|^3 = 1
\imp |f(z)| > |g(z)|$$
The roots of $f$ are
$$0 = f(z) = 8z^2 - 8z + 2 = (4z-2)(2z-1)
\imp z=\frac12,\frac12 \in D$$
so by Rouche's theorem $h$ has 2 roots in $D$.


\item Write
$$h(z) = g(z) + f(z),
\quad g(z) = e^z,
\quad f(z) = 3z$$
On $\ptl D$, $|z|=1$, so write $z=\cis t$.
$$|f(z)| = 3|z| = 3,
\quad |g(z)| = |e^z| = e^{\cos t} \le e^1 < 3
\imp |f(z)| > |g(z)|$$
In $D$, $f$ has a root at $z=0$ of order 1, so by Rouche's theorem $h$ has 1 root in $D$.

\end{enumerate}
	
\end{document}